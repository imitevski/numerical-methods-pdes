
% Default to the notebook output style

    


% Inherit from the specified cell style.




    
\documentclass[11pt]{article}

    
    
    \usepackage[T1]{fontenc}
    % Nicer default font (+ math font) than Computer Modern for most use cases
    \usepackage{mathpazo}

    % Basic figure setup, for now with no caption control since it's done
    % automatically by Pandoc (which extracts ![](path) syntax from Markdown).
    \usepackage{graphicx}
    % We will generate all images so they have a width \maxwidth. This means
    % that they will get their normal width if they fit onto the page, but
    % are scaled down if they would overflow the margins.
    \makeatletter
    \def\maxwidth{\ifdim\Gin@nat@width>\linewidth\linewidth
    \else\Gin@nat@width\fi}
    \makeatother
    \let\Oldincludegraphics\includegraphics
    % Set max figure width to be 80% of text width, for now hardcoded.
    \renewcommand{\includegraphics}[1]{\Oldincludegraphics[width=.8\maxwidth]{#1}}
    % Ensure that by default, figures have no caption (until we provide a
    % proper Figure object with a Caption API and a way to capture that
    % in the conversion process - todo).
    \usepackage{caption}
    \DeclareCaptionLabelFormat{nolabel}{}
    \captionsetup{labelformat=nolabel}

    \usepackage{adjustbox} % Used to constrain images to a maximum size 
    \usepackage{xcolor} % Allow colors to be defined
    \usepackage{enumerate} % Needed for markdown enumerations to work
    \usepackage{geometry} % Used to adjust the document margins
    \usepackage{amsmath} % Equations
    \usepackage{amssymb} % Equations
    \usepackage{textcomp} % defines textquotesingle
    % Hack from http://tex.stackexchange.com/a/47451/13684:
    \AtBeginDocument{%
        \def\PYZsq{\textquotesingle}% Upright quotes in Pygmentized code
    }
    \usepackage{upquote} % Upright quotes for verbatim code
    \usepackage{eurosym} % defines \euro
    \usepackage[mathletters]{ucs} % Extended unicode (utf-8) support
    \usepackage[utf8x]{inputenc} % Allow utf-8 characters in the tex document
    \usepackage{fancyvrb} % verbatim replacement that allows latex
    \usepackage{grffile} % extends the file name processing of package graphics 
                         % to support a larger range 
    % The hyperref package gives us a pdf with properly built
    % internal navigation ('pdf bookmarks' for the table of contents,
    % internal cross-reference links, web links for URLs, etc.)
    \usepackage{hyperref}
    \usepackage{longtable} % longtable support required by pandoc >1.10
    \usepackage{booktabs}  % table support for pandoc > 1.12.2
    \usepackage[inline]{enumitem} % IRkernel/repr support (it uses the enumerate* environment)
    \usepackage[normalem]{ulem} % ulem is needed to support strikethroughs (\sout)
                                % normalem makes italics be italics, not underlines
    

    
    
    % Colors for the hyperref package
    \definecolor{urlcolor}{rgb}{0,.145,.698}
    \definecolor{linkcolor}{rgb}{.71,0.21,0.01}
    \definecolor{citecolor}{rgb}{.12,.54,.11}

    % ANSI colors
    \definecolor{ansi-black}{HTML}{3E424D}
    \definecolor{ansi-black-intense}{HTML}{282C36}
    \definecolor{ansi-red}{HTML}{E75C58}
    \definecolor{ansi-red-intense}{HTML}{B22B31}
    \definecolor{ansi-green}{HTML}{00A250}
    \definecolor{ansi-green-intense}{HTML}{007427}
    \definecolor{ansi-yellow}{HTML}{DDB62B}
    \definecolor{ansi-yellow-intense}{HTML}{B27D12}
    \definecolor{ansi-blue}{HTML}{208FFB}
    \definecolor{ansi-blue-intense}{HTML}{0065CA}
    \definecolor{ansi-magenta}{HTML}{D160C4}
    \definecolor{ansi-magenta-intense}{HTML}{A03196}
    \definecolor{ansi-cyan}{HTML}{60C6C8}
    \definecolor{ansi-cyan-intense}{HTML}{258F8F}
    \definecolor{ansi-white}{HTML}{C5C1B4}
    \definecolor{ansi-white-intense}{HTML}{A1A6B2}

    % commands and environments needed by pandoc snippets
    % extracted from the output of `pandoc -s`
    \providecommand{\tightlist}{%
      \setlength{\itemsep}{0pt}\setlength{\parskip}{0pt}}
    \DefineVerbatimEnvironment{Highlighting}{Verbatim}{commandchars=\\\{\}}
    % Add ',fontsize=\small' for more characters per line
    \newenvironment{Shaded}{}{}
    \newcommand{\KeywordTok}[1]{\textcolor[rgb]{0.00,0.44,0.13}{\textbf{{#1}}}}
    \newcommand{\DataTypeTok}[1]{\textcolor[rgb]{0.56,0.13,0.00}{{#1}}}
    \newcommand{\DecValTok}[1]{\textcolor[rgb]{0.25,0.63,0.44}{{#1}}}
    \newcommand{\BaseNTok}[1]{\textcolor[rgb]{0.25,0.63,0.44}{{#1}}}
    \newcommand{\FloatTok}[1]{\textcolor[rgb]{0.25,0.63,0.44}{{#1}}}
    \newcommand{\CharTok}[1]{\textcolor[rgb]{0.25,0.44,0.63}{{#1}}}
    \newcommand{\StringTok}[1]{\textcolor[rgb]{0.25,0.44,0.63}{{#1}}}
    \newcommand{\CommentTok}[1]{\textcolor[rgb]{0.38,0.63,0.69}{\textit{{#1}}}}
    \newcommand{\OtherTok}[1]{\textcolor[rgb]{0.00,0.44,0.13}{{#1}}}
    \newcommand{\AlertTok}[1]{\textcolor[rgb]{1.00,0.00,0.00}{\textbf{{#1}}}}
    \newcommand{\FunctionTok}[1]{\textcolor[rgb]{0.02,0.16,0.49}{{#1}}}
    \newcommand{\RegionMarkerTok}[1]{{#1}}
    \newcommand{\ErrorTok}[1]{\textcolor[rgb]{1.00,0.00,0.00}{\textbf{{#1}}}}
    \newcommand{\NormalTok}[1]{{#1}}
    
    % Additional commands for more recent versions of Pandoc
    \newcommand{\ConstantTok}[1]{\textcolor[rgb]{0.53,0.00,0.00}{{#1}}}
    \newcommand{\SpecialCharTok}[1]{\textcolor[rgb]{0.25,0.44,0.63}{{#1}}}
    \newcommand{\VerbatimStringTok}[1]{\textcolor[rgb]{0.25,0.44,0.63}{{#1}}}
    \newcommand{\SpecialStringTok}[1]{\textcolor[rgb]{0.73,0.40,0.53}{{#1}}}
    \newcommand{\ImportTok}[1]{{#1}}
    \newcommand{\DocumentationTok}[1]{\textcolor[rgb]{0.73,0.13,0.13}{\textit{{#1}}}}
    \newcommand{\AnnotationTok}[1]{\textcolor[rgb]{0.38,0.63,0.69}{\textbf{\textit{{#1}}}}}
    \newcommand{\CommentVarTok}[1]{\textcolor[rgb]{0.38,0.63,0.69}{\textbf{\textit{{#1}}}}}
    \newcommand{\VariableTok}[1]{\textcolor[rgb]{0.10,0.09,0.49}{{#1}}}
    \newcommand{\ControlFlowTok}[1]{\textcolor[rgb]{0.00,0.44,0.13}{\textbf{{#1}}}}
    \newcommand{\OperatorTok}[1]{\textcolor[rgb]{0.40,0.40,0.40}{{#1}}}
    \newcommand{\BuiltInTok}[1]{{#1}}
    \newcommand{\ExtensionTok}[1]{{#1}}
    \newcommand{\PreprocessorTok}[1]{\textcolor[rgb]{0.74,0.48,0.00}{{#1}}}
    \newcommand{\AttributeTok}[1]{\textcolor[rgb]{0.49,0.56,0.16}{{#1}}}
    \newcommand{\InformationTok}[1]{\textcolor[rgb]{0.38,0.63,0.69}{\textbf{\textit{{#1}}}}}
    \newcommand{\WarningTok}[1]{\textcolor[rgb]{0.38,0.63,0.69}{\textbf{\textit{{#1}}}}}
    
    
    % Define a nice break command that doesn't care if a line doesn't already
    % exist.
    \def\br{\hspace*{\fill} \\* }
    % Math Jax compatability definitions
    \def\gt{>}
    \def\lt{<}
    % Document parameters
    \title{11\_hyperbolic-2}
    
    
    

    % Pygments definitions
    
\makeatletter
\def\PY@reset{\let\PY@it=\relax \let\PY@bf=\relax%
    \let\PY@ul=\relax \let\PY@tc=\relax%
    \let\PY@bc=\relax \let\PY@ff=\relax}
\def\PY@tok#1{\csname PY@tok@#1\endcsname}
\def\PY@toks#1+{\ifx\relax#1\empty\else%
    \PY@tok{#1}\expandafter\PY@toks\fi}
\def\PY@do#1{\PY@bc{\PY@tc{\PY@ul{%
    \PY@it{\PY@bf{\PY@ff{#1}}}}}}}
\def\PY#1#2{\PY@reset\PY@toks#1+\relax+\PY@do{#2}}

\expandafter\def\csname PY@tok@w\endcsname{\def\PY@tc##1{\textcolor[rgb]{0.73,0.73,0.73}{##1}}}
\expandafter\def\csname PY@tok@c\endcsname{\let\PY@it=\textit\def\PY@tc##1{\textcolor[rgb]{0.25,0.50,0.50}{##1}}}
\expandafter\def\csname PY@tok@cp\endcsname{\def\PY@tc##1{\textcolor[rgb]{0.74,0.48,0.00}{##1}}}
\expandafter\def\csname PY@tok@k\endcsname{\let\PY@bf=\textbf\def\PY@tc##1{\textcolor[rgb]{0.00,0.50,0.00}{##1}}}
\expandafter\def\csname PY@tok@kp\endcsname{\def\PY@tc##1{\textcolor[rgb]{0.00,0.50,0.00}{##1}}}
\expandafter\def\csname PY@tok@kt\endcsname{\def\PY@tc##1{\textcolor[rgb]{0.69,0.00,0.25}{##1}}}
\expandafter\def\csname PY@tok@o\endcsname{\def\PY@tc##1{\textcolor[rgb]{0.40,0.40,0.40}{##1}}}
\expandafter\def\csname PY@tok@ow\endcsname{\let\PY@bf=\textbf\def\PY@tc##1{\textcolor[rgb]{0.67,0.13,1.00}{##1}}}
\expandafter\def\csname PY@tok@nb\endcsname{\def\PY@tc##1{\textcolor[rgb]{0.00,0.50,0.00}{##1}}}
\expandafter\def\csname PY@tok@nf\endcsname{\def\PY@tc##1{\textcolor[rgb]{0.00,0.00,1.00}{##1}}}
\expandafter\def\csname PY@tok@nc\endcsname{\let\PY@bf=\textbf\def\PY@tc##1{\textcolor[rgb]{0.00,0.00,1.00}{##1}}}
\expandafter\def\csname PY@tok@nn\endcsname{\let\PY@bf=\textbf\def\PY@tc##1{\textcolor[rgb]{0.00,0.00,1.00}{##1}}}
\expandafter\def\csname PY@tok@ne\endcsname{\let\PY@bf=\textbf\def\PY@tc##1{\textcolor[rgb]{0.82,0.25,0.23}{##1}}}
\expandafter\def\csname PY@tok@nv\endcsname{\def\PY@tc##1{\textcolor[rgb]{0.10,0.09,0.49}{##1}}}
\expandafter\def\csname PY@tok@no\endcsname{\def\PY@tc##1{\textcolor[rgb]{0.53,0.00,0.00}{##1}}}
\expandafter\def\csname PY@tok@nl\endcsname{\def\PY@tc##1{\textcolor[rgb]{0.63,0.63,0.00}{##1}}}
\expandafter\def\csname PY@tok@ni\endcsname{\let\PY@bf=\textbf\def\PY@tc##1{\textcolor[rgb]{0.60,0.60,0.60}{##1}}}
\expandafter\def\csname PY@tok@na\endcsname{\def\PY@tc##1{\textcolor[rgb]{0.49,0.56,0.16}{##1}}}
\expandafter\def\csname PY@tok@nt\endcsname{\let\PY@bf=\textbf\def\PY@tc##1{\textcolor[rgb]{0.00,0.50,0.00}{##1}}}
\expandafter\def\csname PY@tok@nd\endcsname{\def\PY@tc##1{\textcolor[rgb]{0.67,0.13,1.00}{##1}}}
\expandafter\def\csname PY@tok@s\endcsname{\def\PY@tc##1{\textcolor[rgb]{0.73,0.13,0.13}{##1}}}
\expandafter\def\csname PY@tok@sd\endcsname{\let\PY@it=\textit\def\PY@tc##1{\textcolor[rgb]{0.73,0.13,0.13}{##1}}}
\expandafter\def\csname PY@tok@si\endcsname{\let\PY@bf=\textbf\def\PY@tc##1{\textcolor[rgb]{0.73,0.40,0.53}{##1}}}
\expandafter\def\csname PY@tok@se\endcsname{\let\PY@bf=\textbf\def\PY@tc##1{\textcolor[rgb]{0.73,0.40,0.13}{##1}}}
\expandafter\def\csname PY@tok@sr\endcsname{\def\PY@tc##1{\textcolor[rgb]{0.73,0.40,0.53}{##1}}}
\expandafter\def\csname PY@tok@ss\endcsname{\def\PY@tc##1{\textcolor[rgb]{0.10,0.09,0.49}{##1}}}
\expandafter\def\csname PY@tok@sx\endcsname{\def\PY@tc##1{\textcolor[rgb]{0.00,0.50,0.00}{##1}}}
\expandafter\def\csname PY@tok@m\endcsname{\def\PY@tc##1{\textcolor[rgb]{0.40,0.40,0.40}{##1}}}
\expandafter\def\csname PY@tok@gh\endcsname{\let\PY@bf=\textbf\def\PY@tc##1{\textcolor[rgb]{0.00,0.00,0.50}{##1}}}
\expandafter\def\csname PY@tok@gu\endcsname{\let\PY@bf=\textbf\def\PY@tc##1{\textcolor[rgb]{0.50,0.00,0.50}{##1}}}
\expandafter\def\csname PY@tok@gd\endcsname{\def\PY@tc##1{\textcolor[rgb]{0.63,0.00,0.00}{##1}}}
\expandafter\def\csname PY@tok@gi\endcsname{\def\PY@tc##1{\textcolor[rgb]{0.00,0.63,0.00}{##1}}}
\expandafter\def\csname PY@tok@gr\endcsname{\def\PY@tc##1{\textcolor[rgb]{1.00,0.00,0.00}{##1}}}
\expandafter\def\csname PY@tok@ge\endcsname{\let\PY@it=\textit}
\expandafter\def\csname PY@tok@gs\endcsname{\let\PY@bf=\textbf}
\expandafter\def\csname PY@tok@gp\endcsname{\let\PY@bf=\textbf\def\PY@tc##1{\textcolor[rgb]{0.00,0.00,0.50}{##1}}}
\expandafter\def\csname PY@tok@go\endcsname{\def\PY@tc##1{\textcolor[rgb]{0.53,0.53,0.53}{##1}}}
\expandafter\def\csname PY@tok@gt\endcsname{\def\PY@tc##1{\textcolor[rgb]{0.00,0.27,0.87}{##1}}}
\expandafter\def\csname PY@tok@err\endcsname{\def\PY@bc##1{\setlength{\fboxsep}{0pt}\fcolorbox[rgb]{1.00,0.00,0.00}{1,1,1}{\strut ##1}}}
\expandafter\def\csname PY@tok@kc\endcsname{\let\PY@bf=\textbf\def\PY@tc##1{\textcolor[rgb]{0.00,0.50,0.00}{##1}}}
\expandafter\def\csname PY@tok@kd\endcsname{\let\PY@bf=\textbf\def\PY@tc##1{\textcolor[rgb]{0.00,0.50,0.00}{##1}}}
\expandafter\def\csname PY@tok@kn\endcsname{\let\PY@bf=\textbf\def\PY@tc##1{\textcolor[rgb]{0.00,0.50,0.00}{##1}}}
\expandafter\def\csname PY@tok@kr\endcsname{\let\PY@bf=\textbf\def\PY@tc##1{\textcolor[rgb]{0.00,0.50,0.00}{##1}}}
\expandafter\def\csname PY@tok@bp\endcsname{\def\PY@tc##1{\textcolor[rgb]{0.00,0.50,0.00}{##1}}}
\expandafter\def\csname PY@tok@fm\endcsname{\def\PY@tc##1{\textcolor[rgb]{0.00,0.00,1.00}{##1}}}
\expandafter\def\csname PY@tok@vc\endcsname{\def\PY@tc##1{\textcolor[rgb]{0.10,0.09,0.49}{##1}}}
\expandafter\def\csname PY@tok@vg\endcsname{\def\PY@tc##1{\textcolor[rgb]{0.10,0.09,0.49}{##1}}}
\expandafter\def\csname PY@tok@vi\endcsname{\def\PY@tc##1{\textcolor[rgb]{0.10,0.09,0.49}{##1}}}
\expandafter\def\csname PY@tok@vm\endcsname{\def\PY@tc##1{\textcolor[rgb]{0.10,0.09,0.49}{##1}}}
\expandafter\def\csname PY@tok@sa\endcsname{\def\PY@tc##1{\textcolor[rgb]{0.73,0.13,0.13}{##1}}}
\expandafter\def\csname PY@tok@sb\endcsname{\def\PY@tc##1{\textcolor[rgb]{0.73,0.13,0.13}{##1}}}
\expandafter\def\csname PY@tok@sc\endcsname{\def\PY@tc##1{\textcolor[rgb]{0.73,0.13,0.13}{##1}}}
\expandafter\def\csname PY@tok@dl\endcsname{\def\PY@tc##1{\textcolor[rgb]{0.73,0.13,0.13}{##1}}}
\expandafter\def\csname PY@tok@s2\endcsname{\def\PY@tc##1{\textcolor[rgb]{0.73,0.13,0.13}{##1}}}
\expandafter\def\csname PY@tok@sh\endcsname{\def\PY@tc##1{\textcolor[rgb]{0.73,0.13,0.13}{##1}}}
\expandafter\def\csname PY@tok@s1\endcsname{\def\PY@tc##1{\textcolor[rgb]{0.73,0.13,0.13}{##1}}}
\expandafter\def\csname PY@tok@mb\endcsname{\def\PY@tc##1{\textcolor[rgb]{0.40,0.40,0.40}{##1}}}
\expandafter\def\csname PY@tok@mf\endcsname{\def\PY@tc##1{\textcolor[rgb]{0.40,0.40,0.40}{##1}}}
\expandafter\def\csname PY@tok@mh\endcsname{\def\PY@tc##1{\textcolor[rgb]{0.40,0.40,0.40}{##1}}}
\expandafter\def\csname PY@tok@mi\endcsname{\def\PY@tc##1{\textcolor[rgb]{0.40,0.40,0.40}{##1}}}
\expandafter\def\csname PY@tok@il\endcsname{\def\PY@tc##1{\textcolor[rgb]{0.40,0.40,0.40}{##1}}}
\expandafter\def\csname PY@tok@mo\endcsname{\def\PY@tc##1{\textcolor[rgb]{0.40,0.40,0.40}{##1}}}
\expandafter\def\csname PY@tok@ch\endcsname{\let\PY@it=\textit\def\PY@tc##1{\textcolor[rgb]{0.25,0.50,0.50}{##1}}}
\expandafter\def\csname PY@tok@cm\endcsname{\let\PY@it=\textit\def\PY@tc##1{\textcolor[rgb]{0.25,0.50,0.50}{##1}}}
\expandafter\def\csname PY@tok@cpf\endcsname{\let\PY@it=\textit\def\PY@tc##1{\textcolor[rgb]{0.25,0.50,0.50}{##1}}}
\expandafter\def\csname PY@tok@c1\endcsname{\let\PY@it=\textit\def\PY@tc##1{\textcolor[rgb]{0.25,0.50,0.50}{##1}}}
\expandafter\def\csname PY@tok@cs\endcsname{\let\PY@it=\textit\def\PY@tc##1{\textcolor[rgb]{0.25,0.50,0.50}{##1}}}

\def\PYZbs{\char`\\}
\def\PYZus{\char`\_}
\def\PYZob{\char`\{}
\def\PYZcb{\char`\}}
\def\PYZca{\char`\^}
\def\PYZam{\char`\&}
\def\PYZlt{\char`\<}
\def\PYZgt{\char`\>}
\def\PYZsh{\char`\#}
\def\PYZpc{\char`\%}
\def\PYZdl{\char`\$}
\def\PYZhy{\char`\-}
\def\PYZsq{\char`\'}
\def\PYZdq{\char`\"}
\def\PYZti{\char`\~}
% for compatibility with earlier versions
\def\PYZat{@}
\def\PYZlb{[}
\def\PYZrb{]}
\makeatother


    % Exact colors from NB
    \definecolor{incolor}{rgb}{0.0, 0.0, 0.5}
    \definecolor{outcolor}{rgb}{0.545, 0.0, 0.0}



    
    % Prevent overflowing lines due to hard-to-break entities
    \sloppy 
    % Setup hyperref package
    \hypersetup{
      breaklinks=true,  % so long urls are correctly broken across lines
      colorlinks=true,
      urlcolor=urlcolor,
      linkcolor=linkcolor,
      citecolor=citecolor,
      }
    % Slightly bigger margins than the latex defaults
    
    \geometry{verbose,tmargin=1in,bmargin=1in,lmargin=1in,rmargin=1in}
    
    

    \begin{document}
    
    
    \maketitle
    
    

    
    Text provided under a Creative Commons Attribution license, CC-BY. All
code is made available under the FSF-approved MIT license. (c) Kyle T.
Mandli

    \begin{Verbatim}[commandchars=\\\{\}]
{\color{incolor}In [{\color{incolor}2}]:} \PY{o}{\PYZpc{}}\PY{k}{matplotlib} inline
        \PY{k+kn}{from} \PY{n+nn}{\PYZus{}\PYZus{}future\PYZus{}\PYZus{}} \PY{k}{import} \PY{n}{print\PYZus{}function}
        \PY{k+kn}{import} \PY{n+nn}{numpy}
        \PY{k+kn}{import} \PY{n+nn}{matplotlib}\PY{n+nn}{.}\PY{n+nn}{pyplot} \PY{k}{as} \PY{n+nn}{plt}
\end{Verbatim}


    \hypertarget{hyperbolic-equations---part-ii}{%
\section{Hyperbolic Equations - Part
II}\label{hyperbolic-equations---part-ii}}

    \hypertarget{characteristic-tracing}{%
\subsection{Characteristic Tracing}\label{characteristic-tracing}}

The common way to solve hyperbolic PDEs analytically is by using the
method of characteristics but up until now we really have not tried to
use this theory to construct a numerical method.

    Thinking about the value of the solution at a point
\((x_j, t + \Delta t)\) we know for \(a > 0\) that we look backwards to
the point \((x_j - a \Delta t, t)\) where the solution there informs the
solution at the point of interest.

This works great until we start thinking about a discretized grid where
we only know the solution at time \(t\) at a discrete set of points.

\begin{figure}
\centering
\includegraphics{./images/characteristic_tracing_1.png}
\caption{Characteristic Tracing nu != 1}
\end{figure}

 \[ u_t +au_x=0 \] \[u(x,t) = u_0(x-at) \]

    If the characteristics that intersects with \((x_j, t + \Delta t)\) also
intersects with a point at time \(t\) then we are ok. Usually this will
not be the case unless we specifically choose \(\Delta x\) and
\(\Delta t\) such that this is true. It turns out of course that this
happens if \[
    \frac{a \Delta t}{\Delta x} = 1,
\] exactly the upper bound of our stability results.

\begin{figure}
\centering
\includegraphics{./images/characteristic_tracing_2.png}
\caption{Characteristic Tracing nu == 1}
\end{figure}

Similarly if \(\nu < 1\) then we know that the characteristic will not
hit the grid points exactly. Note also that due to the constraint that
\(|\nu| \leq 1\) that we know that the characteristic cannot pass
\(x_{j-1}\).

We could also instead interpolate between the two values that the
characteristic splits and find the value in question. Show that doing
this using linear interpolation leads to the upwind method.

    For the linear interpolation we know that the intersection is at
\(x_p = x_j - a \Delta t\). The linear interpolant is \[\begin{aligned}
    P_1(x) &= \frac{x - x_{j-1}}{x_{j} - x_{j-1}} U^n_{j} + \frac{x - x_j}{x_{j-1} - x_j} U^n_{j-1} \\
    & = \frac{x - x_{j-1}}{\Delta x} U^n_{j} - \frac{x - x_j}{\Delta x} U^n_{j-1}
\end{aligned}\] so that the value is \[\begin{aligned}
    U^{n+1}_j = P_1(x_j - a \Delta t) &= \frac{x_j - a \Delta t - x_{j-1}}{\Delta x} U^n_{j} - \frac{x_j - a \Delta t - x_j}{\Delta x} U^n_{j-1} \\
    &= \frac{\Delta x - a \Delta t}{\Delta x} U^n_{j} + \frac{a \Delta t}{\Delta x} U^n_{j-1} \\
    &= U^n_{j} - \frac{a \Delta t}{\Delta x} (U^n_{j} - U^n_{j-1}).
\end{aligned}\]

 If \(0 \leq \nu \leq 1\) we can re-write the equation
\[ \frac{\Delta x - a \Delta t}{\Delta x} U^n_{j} + \frac{a \Delta t}{\Delta x} U^n_{j-1} = (1-\nu)u_j^n+\nu u^n_{j-1}\]

    Using a similar technique we can also find the Beam-Warming method with
quadratic interpolation: \[\begin{aligned}
    P_2(x) &= \frac{(x - x_{j-1})(x - x_{j-2})}{(x_{j} - x_{j-1}) (x_{j} - x_{j-2})} U^n_{j} + \frac{(x - x_{j})(x - x_{j-2})}{(x_{j-1} - x_{j}) (x_{j-1} - x_{j-2})} U^n_{j-1} + \frac{(x - x_{j})(x - x_{j-1})}{(x_{j-2} - x_{j}) (x_{j-2} - x_{j-1})} U^n_{j-2} \\
    &=\frac{(x - x_{j-1})(x - x_{j-2})}{2 \Delta x^2} U^n_{j} - \frac{(x - x_{j})(x - x_{j-2})}{\Delta x^2} U^n_{j-1} + \frac{(x - x_{j})(x - x_{j-1})}{2 \Delta x^2} U^n_{j-2} \\
    &=\frac{1}{\Delta x^2} \left[\frac{1}{2} U^n_{j} (x - x_{j-1})(x - x_{j-2}) - U^n_{j-1} (x - x_{j})(x - x_{j-2}) + \frac{1}{2} U^n_{j-2} (x - x_{j})(x - x_{j-1}) \right ]
\end{aligned}\] and finally \[\begin{aligned}
    U^{n+1}_j = P_2(x_j - a \Delta t) &= \frac{1}{\Delta x^2} \left[\frac{1}{2} U^n_{j} (\Delta x - a \Delta t)(2 \Delta x - a \Delta t) - U^n_{j-1} (- a \Delta t)(2 \Delta x - a \Delta t) + \frac{1}{2} U^n_{j-2} (- a \Delta t)(\Delta x - a \Delta t) \right ] \\
    &= \frac{1}{\Delta x^2} \left[\frac{1}{2} U^n_{j} (2 \Delta x^2 - 3 a \Delta t \Delta x + a \Delta t^2) - U^n_{j-1} (-2a \Delta t \Delta x + a^2 \Delta t^2) + \frac{1}{2} U^n_{j-2} (-a \Delta t \Delta x + a^2 \Delta t^2) \right ] \\
    &= U^n_j - \frac{a \Delta t}{2 \Delta x} (3 U^n_j - 4 U^n_{j-1} + U^n_{j-2}) + \frac{a \Delta t^2}{2 \Delta x^2} (U^n_j - 2 U^n_{j-1} + U^n_{j-2})
\end{aligned}\]

    \hypertarget{the-courant-friedrichs-lewy-cfl-condition}{%
\subsection{The Courant-Friedrichs-Lewy (CFL)
condition}\label{the-courant-friedrichs-lewy-cfl-condition}}

One interesting result of our characteristic analysis was that the
stability criteria caused the characteristic intersection with \(t_n\)
to be within the interval \([x_{j-1}, x_j]\) when \(a > 0\). This is
indicative of a more general principle for stability for numerics for
PDEs, due to Courant, Friedrichs, and Lewy and often called the CFL
condition.

 Ruling out stability.

    The stability condition that we have been observing time and time again
\[
    \nu = \left | \frac{a \Delta t}{\Delta x} \right | \leq 1
\] turns out to be a necessary condition for methods developed to solve
the advection equation. The value \(\nu\) is often called the
\emph{Courant number} due to this.

    \hypertarget{domain-of-dependence}{%
\subsubsection{Domain of Dependence}\label{domain-of-dependence}}

To make the more general statement about the CFL condition and the
Courant number we need to talk about what the \emph{domain of
dependence} is for a given PDE. We already know what this is for the
advection equation. We know the solution at \((X, T)\) is
\(u(X, T) = u_0(X - a T)\). The domain of dependence then is \[
    \mathcal{D}(X,T) = \{X - a T\}.
\] Another way to think about this is to consider what points could we
change that would effect the solution at \((X,T)\). In the case of the
advection equation it is one point.

    More generally for other PDEs we might expect the domain of dependence
to be larger than a single point but rather a set of points (as is the
case for systems of advection equations) or an entire interval. The heat
equation is one such equation and has domain of dependence
\(\mathcal{D}(X, T) = (-\infty, \infty)\). In other words all points in
the domain effect all other points at any future time. This type of
equation is also said to have infinite \emph{propagation speed} and is
the case for any parabolic PDE and constitutes another way to classify
more complex PDEs.

One could possibly reject this idea for the heat equation after all the
Green's function for a particular point decays exponentially fast away
from a point but unfortunately is still not fast enough. This is also
the source of the conclusion and physical break down of the diffusion
model, material (or heat) will travel infinitely fast.

 At heat equation we have infinite speed of propagation. The Green's
formula kernal for the heat equation was Gaussian distribution. We are
combining all the points at this time step and use it to get to the next
time step. The domain of dependence here is the whole domain!

    \hypertarget{numerical-domain-of-dependence}{%
\subsubsection{Numerical Domain of
Dependence}\label{numerical-domain-of-dependence}}

A numerical method also has a domain of dependence determined by the
stencil used. For instance the Lax-Friedrichs method has the solution
\(U^n_j\) dependent on the points \(U^{n-1}_{j+1}\), \(U^{n-1}_{j}\),
and \(U^{n-1}_{j-1}\). This is generally true for the three-point
methods we developed earlier including the Lax-Wendroff method.

    We can also trace backwards further in time to see which points
\(U^{n-1}_{j+1}\), \(U^{n-1}_{j}\), and \(U^{n-1}_{j-1}\) depend on to
see a growing cone of dependence.

\begin{figure}
\centering
\includegraphics{./images/characteristic_tracing_3.png}
\caption{Domain of Dependence}
\end{figure}

 The CFL number has to \(\leq 1\). The CFL says that the Domain od
Dependence has to be contained within the domain in the ``cone'' (cannot
be outside of the cone).

    As the grid is refined in both time and space respecting the stability
criteria (the CFL condition) we then might expect that the numerical
domain of dependence might converge to the true one. This is actually
not true for the three-point stencils but in fact a weaker condition
does hold, that the numerical domain of dependence should contain the
PDE's domain of dependence.

    If we say continue to refine our grid with the ratio between
\(\Delta t / \Delta x = r\) then the domain of dependence for the point
\((X,T)\) will fill in the interval \([X - T/ r, X+ T/r]\). Since we
want the computed solution \(U(X,T)\) to converge to the true solution
\(u_0(X - a T)\) we need to require \[
    X - T/r \leq X - a T \leq X + T /r.
\] This basically implies that \(u_0(X - aT)\) lies in the numerical
cone of dependence. This also implies that \(|a| \leq 1 /r\) and
therefore \(|a \Delta t / \Delta x| \leq 1\) again giving us the
familiar stability criteria. This then leads us to the general statement
of the CFL condition.

    The CFL condition can then be summed up in the following necessary
condition: \textgreater{} A numerical method is convergent only if its
numerical domain of dependence contains the domain of dependence
determined from the original PDE as \(\Delta t \rightarrow 0\) and
\(\Delta x \rightarrow 0\).

 For convergence we need accuracy and stability. This says more about
the stability and if it is not stable, then it won't converge.

    \hypertarget{example---upwind-methods}{%
\subsubsection{Example - Upwind
Methods}\label{example---upwind-methods}}

Numerical domain depends on the sign of \(a\) but has a 2 point stencil.
Note that if we pick the wrong direction for the upwinding that as
\(\Delta t\) and \(\Delta x\) go to 0 the point \(X - a T\) will never
lie in the cone of dependence.

 One sided cone (left size of previous picture) for \(a>0\). For \(a<0\)
we would have the right side of the picture.

    \hypertarget{example---heat-equation}{%
\subsubsection{Example - Heat Equation}\label{example---heat-equation}}

We have mentioned already that the true domain of dependence for the
heat equation is the entire domain. How does that work for the heat
equation then, especially with an implicit method?

    This would imply that any 3-point stencil (which was what we had been
using) in fact violates the CFL condition. This is indeed true if we fix
the ratio of \(\Delta t / \Delta x\) but in fact we had a stricter
requirement for the relationship, that
\(\Delta t / \Delta x^2 \leq 1 / 2\). This expands the domain of
dependence as \(\Delta t \rightarrow 0\) fast enough that it will cover
the entire domain.

For implicit methods, such as Crank-Nicholson, the CFL condition is
satisfied for any time step \(\Delta t\) due to the coupling of every
point to every other point.

    \hypertarget{modified-equations}{%
\subsection{Modified Equations}\label{modified-equations}}

Another powerful tool for analyzing numerical methods is the use of
modified equations. This approach is similar to what we used for
deriving local truncation error and reveals more about how we might
expect a given numerical method to perform and what the error might
appear as.

    The basic idea is to find a new PDE that may be solved \textbf{exactly}
by the numerical method. In other words if we had a PDE
\(v_t = f(v, v_x, v_{xx}, \cdots)\) then our approximate solution given
some \(\Delta t\) and \(\Delta x\) would satisfy
\(U^n_j = v(x_j, t_n)\). The question can also be posed ``is there a PDE
that \(U^n_j\) better captures?''. We can answer this question via
Taylor series expansions.

    \hypertarget{example---upwind-method}{%
\subsubsection{Example - Upwind Method}\label{example---upwind-method}}

The upwind method is \[
    U^{n+1}_j = U^n_j - \frac{a \Delta t}{\Delta x} (U^n_j - U^n_{j-1}).
\] assuming \(a > 0\). Assume that we have a function \(v(x,t)\) and an
associated PDE (which we do not know yet) that the upwind method solves
exactly.

    First replace the discrete solution \(U\) with the continuous function
\(v(x,t)\) so that we have \[
    v(x, t + \Delta t) = v(x,t) - \frac{a \Delta t}{\Delta x} (v(x,t) - v(x-\Delta x,t)).
\]

    Using Taylor series we know \[\begin{aligned}
    \left(v + v_t \Delta t + \frac{\Delta t^2}{2} v_{tt} + \frac{\Delta t^3}{6} v_{ttt} + \cdots \right ) - v + \frac{a \Delta t}{\Delta x} \left( v - v + \Delta x v_x + \frac{\Delta x^2}{2} v_{xx} - \frac{\Delta x^3}{6} v_{xxx} + \cdots \right ) = 0\\    
    v_t + \frac{\Delta t}{2} v_{tt} + \frac{\Delta t^2}{6} v_{ttt} + \cdots + a \left( v_x + \frac{\Delta x}{2} v_{xx} - \frac{\Delta x^2}{6} v_{xxx} + \cdots \right ) = 0.
\end{aligned}\] Reorganizing the terms in the equation we have \[
    v_t + a v_x = \frac{1}{2}(a \Delta x v_{xx} - \Delta t v_{tt}) + \frac{1}{6} (a \Delta x^2 v_{xxx} - \Delta t^2 v_{ttt}) + \cdots
\] This is the PDE that \(v\) satisfies.

    We can see here that if \(\Delta t\) and \(\Delta x\) go to zero we can
expect that we will recover the original equation the method was meant
to solve. The dominant terms on the right hand side though give us a
glimpse of the behavior of the solution \(v\) if \(\Delta t\) and
\(\Delta x\) are non-zero. For instance if we consider the
\(\mathcal{O}(\Delta x, \Delta t)\) terms we have the equation \[
    v_t + a v_x = \frac{1}{2}(a \Delta x v_{xx} - \Delta t v_{tt}),
\] an equation that also includes something that looks like the second
order wave equation. We can rewrite this even more explicitly by
differentiating both sides with respect to \(t\) \[
    v_{tt} = -a v_{xt} + \frac{1}{2} (a \Delta x v_{xxt} - \Delta t v_{ttt})
\] and with respect to \(x\) \[
    v_{tx} = -a v_{xx} + \frac{1}{2} (a \Delta x v_{xxx} - \Delta t v_{ttx})
\] which combined leads to \[
    v_{tt} = a^2 v_{xx} + \mathcal{O}(\Delta t).
\]

    Inserting this back into the original expression on the right hand side
we can get rid of the second order derivative in time to find \[
    v_t + a v_x = \frac{1}{2} a \Delta x \left(1 - \frac{a \Delta t}{\Delta x} \right) v_{xx} + \mathcal{O}(\Delta x^2, \Delta t^2)
\] which is an advection-diffusion equation similar to what we saw
before except now explicitly formulated in the continuous case. We can
also say then that the upwind discretization gives a solution to the
above advection-diffusion equation to second-order accuracy.

    So what can we take away from this?\\
- This leading order behavior leads us to believe that the error will be
diffusive in nature. - If \(a \Delta t = \Delta x\), i.e.~the Courant
number \(\nu = 1\), then the exact solution will be recovered. - The
coefficient in front of the diffusion operator is
\(\frac{1}{2} (a \Delta x - a^2 \Delta t)\) which is positive if
\(0 < a \Delta t / \Delta x < 1\), another way to see the stability
criteria.

    \hypertarget{example---lax-wendroff}{%
\subsubsection{Example - Lax-Wendroff}\label{example---lax-wendroff}}

Following the same procedure we can derive the leading order terms (up
to \(\mathcal{O}(\Delta t^2, \Delta x^2)\)) for Lax-Wendroff to find \[
    v_t + a v_x = -\frac{1}{6} a \Delta x^2 \left( 1 - \left(\frac{a \Delta t}{\Delta x}\right)^2 \right) v_{xxx}.
\] We can observe a few things from this modified equation - The
Lax-Wendroff approximates an advection-dispersion equation to third
order. - The dominant error will be dispersive (the third derivative
does this) although this error will be smaller than the diffusive error
from the up-wind method above.

    \hypertarget{an-aside---dispersion}{%
\paragraph{An Aside - Dispersion}\label{an-aside---dispersion}}

Consider the PDE \[
    u_t = u_{xxx}
\] as a Cauchy problem. If we Fourier transform the equation we arrive
at the ODE \[
    \hat{u~}_t(\xi,t) = - i \xi^3 \hat{u~}(\xi, t)
\] which has the solution \[
    \hat{u~}(\xi, t) = \hat{u~}_0(\xi) e^{-i \xi^3 t}.
\]

    Note that this looks like the general solution to an advection problem
in that waves will maintain their amplitude, however each Fourier mode
now propagates at its own speed dependent on its wave number. We can see
this by taking the inverse Fourier transform to find \[
    u(x,t) = \frac{1}{\sqrt{2 \pi}} \int^\infty_{-\infty} \hat{u~}_0(\xi) e^{i\xi(x - \xi^2 t)} d\xi.
\]

    Examining the integrand we can see that the \(\xi\) wave number travels
at the speed \(\xi^2\). In contrast the similar path with the advection
equation leads to \[
    u(x,t) = \frac{1}{\sqrt{2 \pi}} \int^\infty_{-\infty} \hat{u~}_0(\xi) e^{i\xi(x - a t)} d\xi
\] where we clearly see all wave numbers \(\xi\) traveling at the speed
\(a\). This is the essential difference between advection and
dispersion, the components of the solution spread out due to their
different effective speeds.

    We can extend this to more general equations of the form \[
    u_t + a u_x + b u_{xxx} = 0
\] where we can write the solution \[
    u(x,t) = \frac{1}{\sqrt{2 \pi}} \int^\infty_{-\infty} \hat{u~}_0(\xi) e^{i \xi (x - (a - b\xi^2) t)} d\xi.
\] Here we see the speed of the components travel at \(a - b \xi^2\) so
the relative values of \(a\) and \(b\) will determine which effect will
be more dominant.

    Back to the Lax-Wendroff method's modified equations we can write down
the group velocity as \[
    c_g = a - \frac{1}{2} a \Delta x^2 \left(1 - \left( \frac{a \Delta t}{\Delta x} \right )^2 \right) \xi^2.
\] For this particular method \(c_g < a\) for all \(\xi\) and hence the
dispersion error trails the waves as seen in the numerical example.

    We can also retain more terms in the modified equation, if we did this
to fourth order we would find \[
    v_t + a v_x + \frac{1}{6} a \Delta x^2 \left(1 - \left( \frac{a \Delta t}{\Delta x} \right )^2 \right) v_{xxx} + \epsilon v_{xxxx} = 0
\] where \(\epsilon = \mathcal{O}(\Delta x^3 + \Delta t^3)\). We now see
that past the dispersive error we will find hyper-diffusion as the
leading error.

    Dispersion and talking about wave numbers \(\xi\) also brings up another
important consideration. If we were interested in highly oscillatory
waves relative to the grid, i.e.~when \(\xi \Delta x \gg 0\), we may run
into problems representing them on a given grid. For \(\xi \Delta x\)
sufficiently small this is not a problem and the modified equation gives
a reasonable estimate as to the dispersion and therefore the group
velocity. If our expected solution contains waves with
\(\xi \Delta x \gg 0\) then higher order terms may be needed to
correctly represent the solution. Usually we therefore rely on plugging
in the ansatz \[
    u(x,t) = e^{i(\xi x_j - \omega(\xi) t_n)}.
\] This clearly has a relation to von Neumann analysis where we have
replaced \(g(\xi)\) with \(e^{-i \omega(\xi) \Delta t}\).

    \hypertarget{example-beam-warming}{%
\subsubsection{Example: Beam-Warming}\label{example-beam-warming}}

As a contrast to the Lax-Wendroff error behavior consider the modified
equation for the Beam-Warming method which is \[
    v_t + a v_x = \frac{1}{6} a \Delta x^2 \left ( 2- \frac{3 a \Delta t}{\Delta x} + \left(\frac{a \Delta t}{\Delta x} \right)^2 \right ) v_{xxx}.
\] We saw with the numerical example that the dispersion error proceeded
the wave and we now can see why as in this case \(c_g > a\).

    \hypertarget{example-leapfrog}{%
\subsubsection{Example: Leapfrog}\label{example-leapfrog}}

The modified equation for leapfrog leads to some interesting conclusions
as we have some fortunate cancellations. Writing the leapfrog method as
\[
    \frac{v(x, t + \Delta t) - v(x, t - \Delta t)}{2 \Delta t} + a \frac{v(x + \Delta x, t) - v(x - \Delta x, t)}{2 \Delta x} = 0
\] we can observe that the modified equations take the form \[
    v_t + a v_x + \frac{1}{6} a \Delta x^2 \left(1 - \left( \frac{a \Delta t}{\Delta x} \right )^2 \right) v_{xxx} = \epsilon_1 v_{xxxxx} + \epsilon_2 v_{xxxxxxx} + \cdots.
\] It turns out that all even order derivative terms drop out leaving us
only with dispersive error. In fact up to fourth order the leapfrog
discretization solves an advection-dispersion equation. We can also see
now again why leapfrog should be called non-dissipative as there are no
error terms that have even derivatives, i.e.~diffusion is not present.

    As a further exercise we can also compute the exact dispersion relation
of the numerical method (the dispersion relation relates the wave number
\(\xi\) to the phase speed, usually denoted \(\omega(\xi)\)). Plugging
in the familiar ansatz similar to von Neumann analysis
\(e^{i(\xi x_j - \omega t_n)}\) we have \[
    e^{-i\omega \Delta t} = e^{i \omega \Delta t} - \frac{a \Delta t}{\Delta x} \left( e^{i \xi \Delta x} - e^{-i\xi \Delta x}\right)
\] leading to \[
    \sin(\omega \Delta t) = \frac{a \Delta t}{\Delta x} \sin(\xi \Delta x).
\]

    We can also compute the group velocity \(c_g\) from this since \[
    c_g = \frac{\text{d} \omega}{\text{d} \xi} = \frac{a \cos(\xi \Delta x)}{\cos(\omega \Delta t)} = \pm \frac{a \cos(\xi \Delta x)}{\sqrt{1 - \nu^2 \sin^2(\xi \Delta x)}}.
\] Note again what happens if \(\nu = 1\).

    \hypertarget{systems-of-hyperbolic-equations}{%
\subsection{Systems of Hyperbolic
Equations}\label{systems-of-hyperbolic-equations}}

We can extend what we have done so far to systems of (linear) hyperbolic
PDEs of the form \[
    u_t + A u_x = 0
\] with an appropriate initial condition \(u(x,0) = u_0(x)\). Here
\(A \in \mathbb R^{s \times s}\) where \(s\) is the number of equations.

    In this case there is a well-defined way to extend our previous idea of
hyperbolic PDEs as we require \(A\) to be diagonalizable with real
eigenvalues for the system of PDEs to be called hyperbolic. The
consequence of this is that we can write \(A\) as \[
    A = R \Lambda R^{-1}
\] were \(R\) are the eigenvectors with \(\Lambda\) containing the
eigenvalues on its diagonal. These eigenvalues fill in for the value we
saw before as the advective speed \(a\) so these being real and finite
matches well with our previous idea of what a hyperbolic equation should
have, information propagates at a finite speed.

    Although less trivial we can still solve linear hyperbolic systems due
to the decomposition of \(A\). Plugging in the decomposition and
multiplying by \(R^{-1}\) on the right leads to \[
    u_t + R \Lambda R^{-1} u_x = 0 \Rightarrow \\
    R^{-1} u_t + \Lambda R^{-1} u_x = 0.
\]

    Defining the \emph{characteristic variables} as \(w = R^{-1} u\) we can
rewrite the system as a set of decoupled equations with \[
    w_t + \Lambda w_x = 0.
\] We know how to solve these as \(w_p(x,t) = w_p(x - \lambda_p t, 0)\).
The initial conditions in the characteristic variables is \[
    w(x, 0) = R^{-1} u_0(x).
\]

    Transforming back to the original variables we in principle need only to
evaluate \[
    u(x,t) = R w(x,t)
\] however this is not so easy due to the form of the solution in \(w\).
Instead we can write the solution as \[
    u(x,t) = \sum^s_{p=1} w_p(x,t) r_p = \sum^s_{p=1}w_p(x - \lambda_p t, 0) r_p.
\]

We now have \emph{characteristics of the \(p\)th family} which refer to
the \(p\)th group of characteristics determined by the \(p\)th
eigenvalue.

    \hypertarget{numerical-methods}{%
\subsubsection{Numerical Methods}\label{numerical-methods}}

We can extend most of the methods we have discussed thus far to systems
by simply replacing the advective speed \(a\) with the matrix \(A\). For
example the Lax-Wendroff method can be generalized to \[
     U^{n+1}_j = U^n_j - \frac{\Delta t}{2 \Delta x} A (U^n_{j+1} - U^n_{j-1})  + \frac{ \Delta t^2}{2 \Delta x^2} A^2 (U^n_{j+1} - 2 U^n_{j} + U^n_{j-1})
\] provided that the Courant number \(\nu < 1\). Note now we need to be
careful about the Courant number as in general the CFL condition
requires that \[
    \nu = \max_{1 \leq p \leq s} \left| \frac{\lambda_p \Delta t}{\Delta x} \right | < 1
\] All of the centered-based approximations are generally applicable
with this stability criteria.

    The methods we have considered that were one-sided however require a bit
more care unless all the eigenvalues of the matrix \(A\) are either
positive or negative. Instead we must decompose the system into its
characteristic variables, apply the method per equation, and
re-transform back. Generally these types of methods are classified as
\emph{Godunov methods}.

    \hypertarget{boundaries}{%
\subsection{Boundaries}\label{boundaries}}

We have mostly ignored boundaries beyond those that are periodic so
let's turn back to the question of non-periodic boundary conditions. Let
's now consider how to incorporate boundaries to find the methods to
solve initial boundary value problems.

    Consider now the hyperbolic PDE defined by \[
    u_t + a u_x = 0 ~~~~ \Omega = [0, 1] \\
    u(x, 0) = u_0(x)
\] before defining boundary conditions. Due to our domain of dependence
discussion we know a bit about when the boundaries will impact our
solution.

    \begin{figure}
\centering
\includegraphics{./images/characteristics_regions_1.png}
\caption{Characteristic boundaries}
\end{figure}

    For the scalar equation and \(a > 0\) then we know \[
    u(x,t) = \left \{ \begin{aligned}
        u_0(x - a t) & & 0 \leq x - at \leq 1 \\
        g_0(t - x / a) & & \text{otherwise}.
    \end{aligned} \right .
\]

    If we have a system of equations with opposite signs for the speeds we
might have a situation that looked like the following instead
\includegraphics{./images/characteristics_regions_2.png}

    \hypertarget{upwind-for-ibvp}{%
\subsubsection{Upwind for IBVP}\label{upwind-for-ibvp}}

Say we use the appropriate upwind method for \(a > 0\) with the grid
\(\Delta x = 1 / (m + 1)\). Upwind describes all the internal equations
with the condition on the left boundary providing the \(U_0\) value,
hence the method completely specifies the problem. The method is stable
with the same condition as before.

Note that we can no longer directly use von Neumann analysis due to the
new boundary. It still can be useful as a stability tool however the
method of lines analysis can be more useful here.

    Consider again the system of ODEs \[
    U'(t) = A U(t) + g(t)
\] where \[
    A = - \frac{a}{\Delta x} \begin{bmatrix}
        1 \\
        -1 & 1 \\
        & -1 & 1 \\
        & & \ddots & \ddots \\
        & & & -1 & 1
    \end{bmatrix} \quad g(t) = \begin{bmatrix} g_0(t) \frac{a}{\Delta x} \\ 0 \\ \vdots \\ 0 \end{bmatrix}.
\]

    Unfortunately this new matrix, although similar to the one considered
before, has very different properties. This new matrix has eigenvalues
uniformly distributed around the circle with radius \(a / \Delta x\) and
centered at \(z = - a / \Delta x\).

    Why are these changes significant? If we follow our previous analysis we
would conclude that the method is stable if \[
    0 \leq \nu \leq 2
\] which is a bit suspicious. It turns out that this is a necessary
condition (although clearly not sufficient). The problem in our analysis
stems from the fact that \(A\) is highly non-normal and we require
further constraints on the \(\epsilon\)-pseudospectra which again leads
to our more familiar stability constraint.

    \hypertarget{outflow-boundaries}{%
\subsubsection{Outflow Boundaries}\label{outflow-boundaries}}

As was mentioned before, often times a numerical method we would like to
use would require the use of boundary conditions where none should
exist. We saw this with the Lax-Wendroff method where the outflow
boundary points are needed by the stencil. We can specify a
\emph{numerical boundary condition} or \emph{artificial boundary
condition} instead of using a one-sided approximation. The prescription
of numerical boundary conditions is long and the analysis tricky so here
we will relegate ourselves to a couple of illustrative examples

    \hypertarget{example}{%
\paragraph{Example}\label{example}}

Consider the leapfrog method on a finite domain with \(a > 0\) and a
given inflow boundary condition \(g_0(t)\). Say we use the upwind method
on the outflow boundary instead of prescribing a condition. It turns out
doing so will introduce waves with \(\xi \Delta x \approx \pi\) that
will move to the left with speed \(-a\).

    \begin{Verbatim}[commandchars=\\\{\}]
{\color{incolor}In [{\color{incolor}3}]:} \PY{c+c1}{\PYZsh{} Implement Leapfrog for the PDE u\PYZus{}t + u\PYZus{}x = 0 on a finite domain [0, 10]}
        \PY{c+c1}{\PYZsh{} domain}
        \PY{n}{u\PYZus{}true} \PY{o}{=} \PY{k}{lambda} \PY{n}{x}\PY{p}{,} \PY{n}{t}\PY{p}{:} \PY{n}{numpy}\PY{o}{.}\PY{n}{exp}\PY{p}{(}\PY{o}{\PYZhy{}}\PY{l+m+mf}{5.0} \PY{o}{*} \PY{p}{(}\PY{p}{(}\PY{n}{x} \PY{o}{\PYZhy{}} \PY{n}{t} \PY{o}{\PYZhy{}} \PY{l+m+mf}{7.0}\PY{p}{)}\PY{o}{*}\PY{o}{*}\PY{l+m+mi}{2}\PY{p}{)}\PY{p}{)}
        
        \PY{n}{m} \PY{o}{=} \PY{l+m+mi}{100}
        \PY{n}{x} \PY{o}{=} \PY{n}{numpy}\PY{o}{.}\PY{n}{linspace}\PY{p}{(}\PY{l+m+mi}{0}\PY{p}{,} \PY{l+m+mf}{10.0}\PY{p}{,} \PY{n}{m}\PY{p}{)}
        \PY{n}{delta\PYZus{}x} \PY{o}{=} \PY{l+m+mf}{10.0} \PY{o}{/} \PY{p}{(}\PY{n}{m} \PY{o}{\PYZhy{}} \PY{l+m+mi}{1}\PY{p}{)}
        \PY{n}{cfl} \PY{o}{=} \PY{l+m+mf}{0.8}
        \PY{n}{delta\PYZus{}t} \PY{o}{=} \PY{n}{cfl} \PY{o}{*} \PY{n}{delta\PYZus{}x}
        
        \PY{n}{U} \PY{o}{=} \PY{n}{u\PYZus{}true}\PY{p}{(}\PY{n}{x}\PY{p}{,} \PY{l+m+mi}{0}\PY{p}{)}
        \PY{n}{t} \PY{o}{=} \PY{l+m+mf}{0.0}
        \PY{c+c1}{\PYZsh{} Jump start with true\PYZhy{}solution}
        \PY{n}{U\PYZus{}new} \PY{o}{=} \PY{n}{u\PYZus{}true}\PY{p}{(}\PY{n}{x}\PY{p}{,} \PY{n}{t} \PY{o}{+} \PY{n}{delta\PYZus{}t}\PY{p}{)}
        \PY{n}{U\PYZus{}old} \PY{o}{=} \PY{n}{U\PYZus{}new}\PY{o}{.}\PY{n}{copy}\PY{p}{(}\PY{p}{)}
        
        \PY{n}{fig} \PY{o}{=} \PY{n}{plt}\PY{o}{.}\PY{n}{figure}\PY{p}{(}\PY{p}{)}
        \PY{n}{fig}\PY{o}{.}\PY{n}{set\PYZus{}figwidth}\PY{p}{(}\PY{n}{fig}\PY{o}{.}\PY{n}{get\PYZus{}figwidth}\PY{p}{(}\PY{p}{)} \PY{o}{*} \PY{l+m+mi}{2}\PY{p}{)}
        \PY{n}{fig}\PY{o}{.}\PY{n}{set\PYZus{}figheight}\PY{p}{(}\PY{n}{fig}\PY{o}{.}\PY{n}{get\PYZus{}figheight}\PY{p}{(}\PY{p}{)} \PY{o}{*} \PY{l+m+mi}{3}\PY{p}{)}
        \PY{n}{axes} \PY{o}{=} \PY{n}{fig}\PY{o}{.}\PY{n}{add\PYZus{}subplot}\PY{p}{(}\PY{l+m+mi}{3}\PY{p}{,} \PY{l+m+mi}{2}\PY{p}{,} \PY{l+m+mi}{1}\PY{p}{)}
        \PY{n}{axes}\PY{o}{.}\PY{n}{plot}\PY{p}{(}\PY{n}{x}\PY{p}{,} \PY{n}{U}\PY{p}{,} \PY{l+s+s1}{\PYZsq{}}\PY{l+s+s1}{ro}\PY{l+s+s1}{\PYZsq{}}\PY{p}{)}
        \PY{n}{axes}\PY{o}{.}\PY{n}{plot}\PY{p}{(}\PY{n}{x}\PY{p}{,} \PY{n}{u\PYZus{}true}\PY{p}{(}\PY{n}{x}\PY{p}{,} \PY{n}{t}\PY{p}{)}\PY{p}{,}\PY{l+s+s1}{\PYZsq{}}\PY{l+s+s1}{k}\PY{l+s+s1}{\PYZsq{}}\PY{p}{)}
        \PY{n}{axes}\PY{o}{.}\PY{n}{set\PYZus{}ylim}\PY{p}{(}\PY{p}{(}\PY{o}{\PYZhy{}}\PY{l+m+mf}{0.1}\PY{p}{,} \PY{l+m+mf}{1.1}\PY{p}{)}\PY{p}{)}
        \PY{n}{axes}\PY{o}{.}\PY{n}{set\PYZus{}title}\PY{p}{(}\PY{l+s+s2}{\PYZdq{}}\PY{l+s+s2}{t = 0.0}\PY{l+s+s2}{\PYZdq{}}\PY{p}{)}
        
        \PY{n}{t} \PY{o}{+}\PY{o}{=} \PY{n}{delta\PYZus{}t}
        \PY{k}{for} \PY{p}{(}\PY{n}{n}\PY{p}{,} \PY{n}{t\PYZus{}final}\PY{p}{)} \PY{o+ow}{in} \PY{n+nb}{enumerate}\PY{p}{(}\PY{p}{(}\PY{l+m+mi}{10}\PY{o}{*}\PY{n}{delta\PYZus{}t}\PY{p}{,} \PY{l+m+mi}{50} \PY{o}{*} \PY{n}{delta\PYZus{}t}\PY{p}{,} \PY{l+m+mi}{100} \PY{o}{*} \PY{n}{delta\PYZus{}t}\PY{p}{,} \PY{l+m+mi}{200} \PY{o}{*} \PY{n}{delta\PYZus{}t}\PY{p}{,} \PY{l+m+mi}{300} \PY{o}{*} \PY{n}{delta\PYZus{}t}\PY{p}{)}\PY{p}{)}\PY{p}{:}
            \PY{k}{while} \PY{n}{t} \PY{o}{\PYZlt{}} \PY{n}{t\PYZus{}final}\PY{p}{:}
                \PY{n}{U\PYZus{}new}\PY{p}{[}\PY{l+m+mi}{0}\PY{p}{]} \PY{o}{=} \PY{n}{U\PYZus{}old}\PY{p}{[}\PY{l+m+mi}{0}\PY{p}{]} \PY{o}{\PYZhy{}} \PY{n}{delta\PYZus{}t} \PY{o}{/} \PY{n}{delta\PYZus{}x} \PY{o}{*} \PY{p}{(}\PY{n}{U}\PY{p}{[}\PY{l+m+mi}{1}\PY{p}{]} \PY{o}{\PYZhy{}} \PY{n}{u\PYZus{}true}\PY{p}{(}\PY{l+m+mf}{0.0}\PY{p}{,} \PY{n}{t}\PY{p}{)}\PY{p}{)}
                \PY{n}{U\PYZus{}new}\PY{p}{[}\PY{l+m+mi}{1}\PY{p}{:}\PY{o}{\PYZhy{}}\PY{l+m+mi}{1}\PY{p}{]} \PY{o}{=} \PY{n}{U\PYZus{}old}\PY{p}{[}\PY{l+m+mi}{1}\PY{p}{:}\PY{o}{\PYZhy{}}\PY{l+m+mi}{1}\PY{p}{]} \PY{o}{\PYZhy{}} \PY{n}{delta\PYZus{}t} \PY{o}{/} \PY{n}{delta\PYZus{}x} \PY{o}{*} \PY{p}{(}\PY{n}{U}\PY{p}{[}\PY{l+m+mi}{2}\PY{p}{:}\PY{p}{]} \PY{o}{\PYZhy{}} \PY{n}{U}\PY{p}{[}\PY{p}{:}\PY{o}{\PYZhy{}}\PY{l+m+mi}{2}\PY{p}{]}\PY{p}{)}
                \PY{c+c1}{\PYZsh{} Use upwind for outflow boundary}
                \PY{n}{U\PYZus{}new}\PY{p}{[}\PY{o}{\PYZhy{}}\PY{l+m+mi}{1}\PY{p}{]} \PY{o}{=} \PY{n}{U}\PY{p}{[}\PY{o}{\PYZhy{}}\PY{l+m+mi}{1}\PY{p}{]} \PY{o}{\PYZhy{}} \PY{n}{delta\PYZus{}t} \PY{o}{/} \PY{n}{delta\PYZus{}x} \PY{o}{*} \PY{p}{(}\PY{n}{U}\PY{p}{[}\PY{o}{\PYZhy{}}\PY{l+m+mi}{1}\PY{p}{]} \PY{o}{\PYZhy{}} \PY{n}{U}\PY{p}{[}\PY{o}{\PYZhy{}}\PY{l+m+mi}{2}\PY{p}{]}\PY{p}{)}
                \PY{n}{U\PYZus{}old} \PY{o}{=} \PY{n}{U}\PY{o}{.}\PY{n}{copy}\PY{p}{(}\PY{p}{)}
                \PY{n}{U} \PY{o}{=} \PY{n}{U\PYZus{}new}\PY{o}{.}\PY{n}{copy}\PY{p}{(}\PY{p}{)}
                \PY{n}{t} \PY{o}{+}\PY{o}{=} \PY{n}{delta\PYZus{}t}
        
            \PY{c+c1}{\PYZsh{} Plot solution at t = 17.0 and t = 0.0}
            \PY{n}{axes} \PY{o}{=} \PY{n}{fig}\PY{o}{.}\PY{n}{add\PYZus{}subplot}\PY{p}{(}\PY{l+m+mi}{3}\PY{p}{,} \PY{l+m+mi}{2}\PY{p}{,} \PY{n}{n} \PY{o}{+} \PY{l+m+mi}{2}\PY{p}{)}
            \PY{n}{axes}\PY{o}{.}\PY{n}{plot}\PY{p}{(}\PY{n}{x}\PY{p}{,} \PY{n}{U}\PY{p}{,} \PY{l+s+s1}{\PYZsq{}}\PY{l+s+s1}{ro}\PY{l+s+s1}{\PYZsq{}}\PY{p}{)}
            \PY{n}{axes}\PY{o}{.}\PY{n}{plot}\PY{p}{(}\PY{n}{x}\PY{p}{,} \PY{n}{u\PYZus{}true}\PY{p}{(}\PY{n}{x}\PY{p}{,} \PY{n}{t}\PY{p}{)}\PY{p}{,}\PY{l+s+s1}{\PYZsq{}}\PY{l+s+s1}{k}\PY{l+s+s1}{\PYZsq{}}\PY{p}{)}
            \PY{n}{axes}\PY{o}{.}\PY{n}{set\PYZus{}ylim}\PY{p}{(}\PY{p}{(}\PY{o}{\PYZhy{}}\PY{l+m+mf}{0.1}\PY{p}{,} \PY{l+m+mf}{1.1}\PY{p}{)}\PY{p}{)}
            \PY{n}{axes}\PY{o}{.}\PY{n}{set\PYZus{}title}\PY{p}{(}\PY{l+s+s2}{\PYZdq{}}\PY{l+s+s2}{t = }\PY{l+s+si}{\PYZpc{}s}\PY{l+s+s2}{\PYZdq{}} \PY{o}{\PYZpc{}} \PY{n}{t\PYZus{}final}\PY{p}{)}
        
        \PY{n}{plt}\PY{o}{.}\PY{n}{show}\PY{p}{(}\PY{p}{)}
\end{Verbatim}


    \begin{center}
    \adjustimage{max size={0.9\linewidth}{0.9\paperheight}}{output_58_0.png}
    \end{center}
    { \hspace*{\fill} \\}
    
    In general dealing with outflow boundary conditions is very difficult.
Often we will instead of prescribing a one-sided method want to specify
the numerical boundaries which have special properties, such as being
non-reflective or absorbing (we see a slight reflected wave in the above
example).

    \hypertarget{alternatives}{%
\subsection{Alternatives}\label{alternatives}}

Finally we end this discussion with a few alternatives not mentioned
above.

    \hypertarget{higher-order-discretizations}{%
\subsubsection{Higher Order
Discretizations}\label{higher-order-discretizations}}

We can of course use arbitrarily high order discretizations beyond what
we talked about above by employing the method of lines and discretizing
space and time \[
    U_j'(t) = -a W_j(t)
\] assuming the solution remains sufficiently smooth. One example could
be \[
    W_j(t) = \frac{4}{3} \left(\frac{U_{j+1} - U_{j-1}}{2 \Delta x} \right )- \frac{1}{3} \left(\frac{U_{j+2} - U_{j-2}}{4 \Delta x} \right ).
\] The finite difference discretizations discussed so far can all be
used but the higher-order accuracy we reach comes at the cost of a wider
the stencil which leads to difficulty for the usual reasons.

    Another approach to avoid this is to use compact differencing methods
which solve linear systems. A simple example of this idea is \[
    \frac{1}{4} W_{j-1} + W_j + \frac{1}{4} W_{j+1} = \frac{3}{2} \left( \frac{U_{j+1} - U_{j-1}}{2 \Delta x} \right )
\] which leads to a \(\mathcal{O}(\Delta x^4)\) approximation.

    \hypertarget{spectral-methods}{%
\subsubsection{Spectral Methods}\label{spectral-methods}}

We can also use spectral methods to transform the spatial derivatives
into a linear system. In essence we can derive a dense differentiation
matrix \(D\) so that \(W = D U\). These can easily be generalized to
more complex systems of equations but require smooth solutions to work
and can be very difficult to analyze.

    \hypertarget{other-time-discretizations}{%
\subsubsection{Other Time
Discretizations}\label{other-time-discretizations}}

We can of course also use different time discretizations. Above we used
what looked like forward Euler and leapfrog but you can use
higher-order, explicit methods such as Runge-Kutta methods or an
implicit method. Implicit methods can be useful if you are not as
concerned about accuracy but want to evolve the solution to large times.
Also, although the advection equation itself is not stiff, some
hyperbolic PDEs can be or the spatial discretization can as well (as is
the case with the spectral approach above).

    \hypertarget{conservation-laws-and-finite-volume-methods}{%
\subsubsection{Conservation Laws and Finite Volume
Methods}\label{conservation-laws-and-finite-volume-methods}}

A large and important class of hyperbolic PDEs are conservation laws of
the form \[
    u_t + f(u)_x = 0.
\]

    These naturally arise in many areas of physics and describe the
evolution of quantities such as mass, momentum, or energy. One such
system is the Euler equations describing compressible gas dynamics:
\[\begin{aligned}
    &\rho_t + (\rho u)_x = 0 \\
    &(\rho u)_t + (\rho u^2 + p)_x = 0 \\
    &(E)_t + [(E + p) u]_x = 0
\end{aligned}\] describing density \(\rho\), momenta \(\rho u\) and
energy \(E\) coupled with an appropriate equation of state relating
pressure, density, and internal energy.

    A more natural way to formulate conservation laws is with integral forms
of the same equations. In general we can write these as \[
    \frac{\text{d}}{\text{d}t} \int^{x_2}_{x_1} u(x, t) dx = f(u(x_1,t)) - f(u(x_2, t)).
\]

    Methods for solving these often evolve cell averages of \(u\) rather
than point values. In this case our approximation \(U^n_i\) is viewed as
this average over a grid cell \([x_{i-1/2}, x_{i+1/2}]\) with length
\(\Delta x\) and centered at \(x_i\). The cell average would then be \[
    U^n_i \approx \frac{1}{\Delta x} \int^{x_{i+1/2}}_{x_{i-1/2}} u(x, t_n) dx.
\] Finite volume methods generally take this approach with the
specification of a way to evaluate the flux functions the primary
numerical goal.


    % Add a bibliography block to the postdoc
    
    
    
    \end{document}
